\errorcontextlines 999999
\usepackage[ngerman]{babel}

\usepackage[%
    sopra-listings={encoding,cpalette,highlights},%
    sopra-tables, color-palettes={addons},%
    lecture-bibliography={biber,style=numeric-comp},%
    util, lithie-boxes={germanenv,koma,overwrite},%
    lithie-task-boxes={cpalette}, lecture-links={patchurl},%
    lecture-registers={disable}% would interfere with beamer
]{lithie-util}

\makeatletter
\sol@list@define@styles{%
  {keywordA: \@declaredcolor{sol@colors@lst@keywordA}\bfseries},%
}
\makeatother

\RestyleAlgo{plain}
\lstset{lineskip=5.5pt}
\lstfs{10}

\DefinePalette{Rekursion}
{Red,rötlich: RGB(93, 46, 70)}
{Blau,bläulich: RGB(22, 105, 122)}
{Lila,lilafarben: RGB(93, 45, 108)}
{Grün,grünlich: RGB(21, 150, 90)}
\SetShadeContrast{45}
\UsePalette{Rekursion}

\usetheme[libs,nofootfade,centerfoot]{dividing-lines}
\SetColorProfile*{paletteA}{paletteB}{paletteC}

\usetikzlibrary{arrows.meta,decorations,decorations.pathreplacing,shapes.multipart,tikzmark}

\def\info#1{\bgroup\scriptsize\textcolor{gray}{(#1)}\egroup}
\SetAllLinkStyle{#1}
\def\fillfont{\def\mdseries@sf{medium}\sffamily}
\colorlet{lgray}{lightgray!48!white}
\tikzset{
    ldesc/.style={gray,font=\sffamily\sbfamily},
    lrel/.style={fill=white,rounded corners,minimum width=28mm,minimum height=7.5mm,align=center},
    lrel2/.style={fill=white,rounded corners,minimum width=28mm,minimum height=7.5mm*2,align=center},
    lsf/.style={fill=white,rounded corners,minimum width=28mm,minimum height=7.5mm*2,align=center,
        rectangle split, rectangle split parts=2},
    blob/.style={circle,draw, minimum size=1.9em,align=center},
    lblob/.style={blob,writ,font=\fillfont,#1},%
    lblob/.default={fill=lgray, draw=lgray},
    every picture/.append style={line join=round,line cap=round},%
    every node/.append style={font=\sffamily},%
    lblock/.style={block,writ,font=\fillfont,#1},%
    lblock/.default={fill=lgray, draw=lgray},
    block/.style={rectangle,draw, align=center, minimum height=1.25em,rounded corners=1.55pt},%
}

\newcommand\parallelcontent[3][t]{%
    \begin{columns}[#1]
    \begin{column}{.475\linewidth}#2\end{column}\hfill
    \begin{column}{.475\linewidth}#3\end{column}
    \end{columns}
}

\usepackage[glows]{tikzpingus}
\usetikzlibrary{decorations.text,graphs}
\hypersetup{colorlinks=false}

\title{Die Liebe zur Rekursion}
\subtitle{Der Java-Stack, Funktionsaufrufe und Rekursion.\\Ein wiederkehrendes Dilemma.}
\institute{SP, Universität Ulm}

\author{Florian Sihler}
\email{florian.sihler@uni-ulm.de}

\date{\today}
\outro{Ulm, \today}
\license[]{All Rights reserved}

\def\PreTitlepage{\begingroup%
\let\oldinserttitle\inserttitle% allow it to be white on second slide
\let\oldinsertsubtitle\insertsubtitle% allow it to be white on second slide
\colorlet{PINGU@WHITE}{pingu@white}% hacksies for the whites
\only<2|handout:2>{\def\inserttitle{\color{pingu@white}\oldinserttitle}\def\insertsubtitle{\textcolor{pingu@white}{\oldinsertsubtitle}}}%
\onslide<2|handout:2>{%
\savebox0{\tikz{\pingu[lightsaber left=paletteC!90!white,lightsaber left angle=-20,eyes=shiny,right wing grab,cup=paletteA,body=paletteA!20!black,lightsaber left outer glow factor=.11]}}%
\begin{tikzpicture}[overlay,remember picture]%
    % beamer does not support changes of full background easily. so we do hacksies
    \pgfonlayer{background}
    \path[fill,black!99] (current page.north west) rectangle (current page.south east);
    \endpgfonlayer
    \node[above right=.15cm,scale=.65,xshift=.15cm] (pingu) at(current page.south west) {\usebox0};
    \node[below right,pingu@white,text width=.6\paperwidth,align=flush left] at(pingu.north east){Officially supported by the Pingu-Foundation for Emotional Support. There will be light and there will be more peeps to come!};
\end{tikzpicture}}%
}
\def\PostTitlepage{\endgroup}

\addbibresource{./references.bib}

\makeatletter
\newcommand*\md{\@ifstar{\@md}{\@md{0}}}% with star we can set handout state
\def\@md#1#2{\only<#2|handout:#1>{\llap{\color{shadeA}\textbullet~}}}
\newcommand*\mb[2][0]{\only<#2|handout:#1>{\rlap{\smash{\raisebox{-.66\baselineskip}{\color{shadeA}\textbullet}}}}}
\newcommand*\mh[2][0]{\only<#2|handout:#1>{\color{shadeA}\textbullet}}
\makeatother

\setcounter{tocdepth}{4}
\newsavebox\pinguA \newsavebox\pinguB \newsavebox\pinguC

\usepackage[link]{qrcode}
\outroright{\smash{\raisebox{1.33cm/2}{\qrcode[height=1.33cm]{https://github.com/EagleoutIce/Episode-Recursion}}}\begin{tikzpicture}[remember picture,overlay]
    \node[above left,btdl@color@white,scale=.475] at (current page.south east) {\href{https://github.com/EagleoutIce/Episode-Recursion}{Slides on GitHub!}};
\end{tikzpicture}}
